\documentclass[]{article}
\usepackage{lmodern}
\usepackage{amssymb,amsmath}
\usepackage{ifxetex,ifluatex}
\usepackage{fixltx2e} % provides \textsubscript
\ifnum 0\ifxetex 1\fi\ifluatex 1\fi=0 % if pdftex
  \usepackage[T1]{fontenc}
  \usepackage[utf8]{inputenc}
\else % if luatex or xelatex
  \ifxetex
    \usepackage{mathspec}
    \usepackage{xltxtra,xunicode}
  \else
    \usepackage{fontspec}
  \fi
  \defaultfontfeatures{Mapping=tex-text,Scale=MatchLowercase}
  \newcommand{\euro}{€}
\fi
% use upquote if available, for straight quotes in verbatim environments
\IfFileExists{upquote.sty}{\usepackage{upquote}}{}
% use microtype if available
\IfFileExists{microtype.sty}{%
\usepackage{microtype}
\UseMicrotypeSet[protrusion]{basicmath} % disable protrusion for tt fonts
}{}
\usepackage[margin=1in]{geometry}
\usepackage{longtable,booktabs}
\usepackage{graphicx}
\usepackage{subfigure}
\makeatletter
\def\maxwidth{\ifdim\Gin@nat@width>\linewidth\linewidth\else\Gin@nat@width\fi}
\def\maxheight{\ifdim\Gin@nat@height>\textheight\textheight\else\Gin@nat@height\fi}
\makeatother
% Scale images if necessary, so that they will not overflow the page
% margins by default, and it is still possible to overwrite the defaults
% using explicit options in \includegraphics[width, height, ...]{}
\setkeys{Gin}{width=\maxwidth,height=\maxheight,keepaspectratio}
\ifxetex
  \usepackage[setpagesize=false, % page size defined by xetex
              unicode=false, % unicode breaks when used with xetex
              xetex]{hyperref}
\else
  \usepackage[unicode=true]{hyperref}
\fi
\hypersetup{breaklinks=true,
            bookmarks=true,
            pdfauthor={},
            pdftitle={4\^{}\{th\} Report},
            colorlinks=true,
            citecolor=blue,
            urlcolor=blue,
            linkcolor=magenta,
            pdfborder={0 0 0}}
\urlstyle{same}  % don't use monospace font for urls
\setlength{\parindent}{0pt}
\setlength{\parskip}{6pt plus 2pt minus 1pt}
\setlength{\emergencystretch}{3em}  % prevent overfull lines
\setcounter{secnumdepth}{5}

%%% Use protect on footnotes to avoid problems with footnotes in titles
\let\rmarkdownfootnote\footnote%
\def\footnote{\protect\rmarkdownfootnote}

%%% Change title format to be more compact
\usepackage{titling}

% Create subtitle command for use in maketitle
\newcommand{\subtitle}[1]{
  \posttitle{
    \begin{center}\large#1\end{center}
    }
}

\setlength{\droptitle}{-2em}
  \title{Report}
  \pretitle{\vspace{\droptitle}\centering\huge}
  \posttitle{\par}
  \author{}
  \preauthor{}\postauthor{}
  \predate{\centering\large\emph}
  \postdate{\par}
  %\date{7th of July}



\begin{document}

%\maketitle

\vspace*{-2cm}
\section*{General way for a set of trees}

Let $S=(s_1,...,s_m)$ be a set of trees and $ \mathcal{L} = (l_1(\theta),...,l_m(\theta))$ the set of log-likelihood functions of $S$. \\

Then 

$$ l_j(\theta) = \displaystyle\sum_{i=1}^{N_j} - \sigma_{i,j} t_{i,j}+log(\rho_{i,j}) $$
where $N_j$ is the number of branching times of the $j-$tree, $t_{i,j}$ is the $i^{th}$ branching time of the $j$-tree and $\sigma_{i,j}$ and $\rho_{i,j}$ are functions of $\lambda_{i,j}(\theta)$ and $\mu_{i,j}(\theta)$, which are the speciation and extinction rates of the species of the tree $j$ at time $t_{i,j}$ as described in previous reports.\\

In order to solve the E-step on the EM rutine, under the monte-carlo approach, we need to calculate 

$$ l(\theta) = \displaystyle\sum_{j = 1}^m l_j(\theta) $$

the M-step corresponds to find $\max_{\theta} l(\theta) $.

\subsection*{Diversity dependence model}

 
As described previously, we define 

$$ l_j(\theta) = \text{log-likelihood}(\theta | s_j) $$

in the case of diversity-dependence, where we have \\

$\sigma_{i,j} = n_{i,j}\lambda -\beta n_{i,j}^2+n_{i,j}\mu$ \\
$\rho_{i,j} = E_{i,j}(\lambda-\beta n_{i,j})+(1-E_{i,j})\mu$ \\

Thus,

$$ l_j((\lambda,\beta,\mu)) = \displaystyle\sum_{i=1}^{N_j} -t_{i,j}[n_{i,j}\lambda-n_{i,j}^2 \beta - n_{i,j}\mu]+log(E_{i,j}(\lambda-\beta n_{i,j})+(1-E_{i,j})\mu)$$

Moreover, 

$$ l(\theta) = \displaystyle\sum_{j=1}^m \displaystyle\sum_{i=1}^{N_j} -t_{i,j}[n_{i,j}\lambda -n_{i,j}^2 \beta + n_{i,j} \mu] + log(E_{i,j}(\lambda-\beta n_{i,j})+(1-E_{i,j})\mu) $$ 

where, as in the case of 1 single tree, we can find an analytical solution for the parameter $\mu$

$$ \mu = \frac{\displaystyle\sum_{j=1}^m \displaystyle\sum_{i=1}^{N_j}(1-E_{i,j})}{\displaystyle\sum_{j=1}^m \displaystyle\sum_{i=1}^{N_j} t_{i,j}n_{i,j}} $$

and the other two parameters might be calculated under the same framework of the single tree case. 


\end{document}
