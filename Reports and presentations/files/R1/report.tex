\documentclass[a4paper,9pt]{extarticle}
\usepackage[utf8]{inputenc}


\usepackage[a4paper, total={7in, 13in}]{geometry}



%opening
\title{}
\author{}

\begin{document}

\maketitle

%\begin{abstract}

%\end{abstract}


\section*{Readings}

\begin{itemize}
  \item Ecology
    \begin{itemize}
      \item Biological modeling: \cite{bulmer} , \cite{hilborn} (data applications).
      \item Phylogenies and birth-death models: \cite{etienne}, \cite{nee}.
    \end{itemize}
 \item Math
    \begin{itemize}
      \item Stochastic differential equations modeling: \cite{wilkinson}, \cite{pellin} (applied to cell differentiation). 
      \item GLM: \cite{dobson}.
      \item DGlars: \cite{augugliaro}.
    \end{itemize}
    
\end{itemize}

\section*{Some first steps}

  We define \begin{itemize}
  
\item  $X_t$: Number of species on time $t$. 
  
\item  $\lambda$: Speciation rate.
  
\item  $\delta$ : Extinction rate.
  
\item  $T_i$: Time when the event $i$ (speciation or extinction) occurs.
  
\item  $T^{(\lambda,j)}$: Random variable corresponding to the time when an speciation occurs after the event $j-1$.
  
\item  $T^{(\delta,j)}$: Random variable corresponding to the time when an extinction occurs after the event $j-1$.

\end{itemize}

 We assume that $\lambda$ and $\delta$ are constant. We also assume that any species have same probability to speciate 
 and same probability to get extinct, moreover for any $j \in \{1,2,...,X_{j-1}\}$ we have $$T^{(\lambda,j)} \sim exp(\lambda)$$ and $$T^{(\delta,j)} \sim exp(\delta)$$ then $$T_i \sim exp(X_{T_{i-1}}(\lambda+\delta))$$
  
%\subsection*{Some Asumptions}

%\begin{itemize}
% \item $\lambda$ and $\delta$ are constants.
% \item All species have same probability to speciate and same probability to get extinct, moreover $T^{(\lambda,j)} \sim exp(\lambda)$ and $T^{(\delta,j)} \sim exp(\delta)$ for all $j$.
% \end{itemize}

%\subsection*{The first objective}


Given the whole phylogenic tree (data), we are interested in estimate $\lambda$ and $\delta$. For that, three methods are sugested: 

\begin{itemize}
 \item MLE
 \item Bayesian inference
 \item Method of moments
\end{itemize}

%\subsection*{MLE}



%Note that $T_i = \min $ ...  then $T_i \sim exp(X_{T_{i-1}(\lambda+\delta)})$
  
 
\begin{thebibliography}{widest entry}
\bibitem[HilMan]{hilborn} R. Hilborn and M. Mangel. \emph{The Ecological Detective, Confronting models with data}, 1997.
\bibitem[Bul]{bulmer} M. Bulmer. \emph{Theoretical Evolutionary Ecology}, 1994.
\bibitem[Wil]{wilkinson}D. J. Wilkinson. \emph{Stochastic Modelling for Systems Biology}, 2006.
\bibitem[DobBar]{dobson} A. J. Dobson, A.G. Barnett \emph{An Introduction to Generalized Liner Models}, 2008.
\bibitem[Etienne]{etienne} R. S. Etienne et al. \emph{Diversity-dependence brings molecular phylogenies closer to agreement with the fossil record}, 2011.
\bibitem[Nee]{nee} S. Nee. \emph{Birth-Death Models in Macroevolution}, 2006.
\bibitem[Pellin]{pellin} P. Pellin et al. \emph{A stochastic model for cell differentiation, efficient parameters estimation and model selection}, 2016.
\bibitem[Augugliaro]{augugliaro} L. Augugliaro, A.M. Mineo and E.C Wit. \emph{Differential geometrix least angle regression: a differential geometric approach to sparse generalized models}, 2012.

\end{thebibliography}

\end{document}
