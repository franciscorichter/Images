\chapter{Knowledge utilization}




\section{Implementation of results}


\subsection{Software development}
We will develop free software packages for the often mathematically and computationally demanding inference techniques, allowing non-specialist scientist to apply these newly developed methods easily to their own systems.


\subsection{Application to real data}

We will test the methods on published phylogenies, but also on a new phylogeny of microlandsnails on limestone outcrops on Malaysian Borneo that will be constructed in research funded by a VICI grant awarded to R.S. Etienne. This system is unique in that contains well-defined local communities (the snails are strongly restricted to the limestone) with limited dispersal and high endemism. Besides contituting a miniature world ideally suited for trying out our new methods, the microsnail ecological evolutionary dynamica is a real life system that can reveal the effects of global warming on biodiversity.

\section{Contribution of research to specific fields beyond mathematics, ecology and evolution: Language evolution}

Phylogenetic methods are increasingly used to study language evolution. This project can help identify the phylogeographic context and the influence of community effects (e.g number of languages present) on language diversification in different human societies. We will collaborate with Quentin Atkinson (University of Auckland) and localized evolution models are also there an important inferential hurdle that has not been taken yet.

\section{Contribution of research to society: Global conservation} 

Understanding the processes driving biodiversity is of crucial importance for assessing the effects of global change on the diversity of life on this planet. This project will help addressing the following key issues: (1) if local diversity limits further diversification, will a reduction in diversity then speed up diversification and restore the balance, and if so, over what time scale will equilibrium be recovered; or (2) will lost diversity never be regained, because the loss happens too quickly for natural processes to compensate?

\section{Open Access policy}

Scientific publications arising from this project will be made publicly accessible to the research community, by depositing submitted/accepted manuscripts at arXiv.org and choosing Open Access journals options. This will allow the unconditional and immediate availability of the scientific results for use by other scientist and general public, particularly in developing countries. 